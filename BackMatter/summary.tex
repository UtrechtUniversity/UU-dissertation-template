%%%%%%%%%%%%%%%%%%%%%%%%%%%%%%%%%%%%%%%%%%%%%%%%%%
%%%%		~~~~ EDITOR'S NOTE ~~~~
%%
%% This file contains an a base fot the English Summary
%% - A summary in Dutch is required if the dissertation is not
%% - A summary in English is often included
%% 
%% This is the PhD thesis of Laurens P. Stoop in LaTeX
%%                       - Comments, remarks and more: email me at laurensstoop@protonmail.com
%%
%%%%%%%%%%%%%%%%%%%%%%%%%%%%%%%%%%%%%%%%%%%%%%%%%%

%% Name
\chapter{English summary}


%%%%%%%%%%%%%%%%%%%%%%%%%%%%%%%%%%%%%%%%%%%%%%%%%%
%%%% ~~~~ Introduction
% Intro CC -> energy transition
Humanity is changing Earth's climate.
Immediate and strong mitigation of climate change is required across all sectors of society, otherwise we will miss the brief and rapidly closing window to secure a liveable future.
Within the energy sector, a transition towards sustainable resources is already starting to take shape and accelerate.
This transition increases the impact weather can have on the energy system operations and introduces an additional source of variability in energy systems.
Knowledge on the emerging compound energy-meteorological variability is required to support the policy makers, energy system planners, and operators around the world that are guiding the energy transition.


%%%%%%%%%%%%%%%%%%%%%%%%%%%%%%%%%%%%%%%%%%%%%%%%%%
%%%% ~~~~ Part I Challenges and RQ

\subsection*{Research objective}
% Aim of thesis
This thesis aims to contribute to tackling some of the open research challenges in this field by providing data driven insight into the relevant impact of energy-meteorological variability on energy system operation, specifically for the use of energy system planners and policymakers.
To do this, techniques from algorithmic data analysis are combined with knowledge on energy and resources, as well as an understanding of the driving physical forces of the weather and climate.

% structure
This thesis consists of two parts: the first delves into data driven methods and measures to gain a better understanding of energy-meteorological variability; the second discusses methods to integrate this understanding in energy system operations.


%%%%%%%%%%%%%%%%%%%%%%%%%%%%%%%%%%%%%%%%%%%%%%%%%%
%%%% ~~~~ Part II Metrics and measures
\subsection*{Data driven metrics and measures}
% Part 1: Data driven methods
In the first part, three papers are presented that aim to determine and apply relevant data driven methods and approaches to quantify and understand the energy-meteorological variability present in future power systems and the critical, high impact due to this variability.

% CS1
The first paper discusses an outlier detection technique, i.e. the Maximum Divergent Intervals algorithm, to systematically find critical, high impact events that are relevant for energy system reliability.
Furthermore, it discusses the properties of the events found over a 70-year historical period.

% CP2
The second paper develops the Climatological Renewable Energy Deviation Index that can be used to quantify energy-meteorological variability across timescales, and discusses its use for researchers and stakeholders to assess the impact of energy-meteorological variability in energy system operation.

% MSc1
The third paper contains an investigation into extreme impacts of energy-meteorological variability on a future energy system using a large -- 3$\times$2000 years -- climate simulation dataset, and presents a characterisation of the 1-, 7-, and 14-day duration events found.

%%%%%%%%%%%%%%%%%%%%%%%%%%%%%%%%%%%%%%%%%%%%%%%%%%
%%%% ~~~~ Part III towards integration
\subsection*{Towards integrating the understanding in operation}
% Part 2: Towards understanding
In the second part, two papers are presented that aim to integrate the understanding, and the representation, of energy-meteorological variability with operational energy system models to address knowledge gaps for those guiding the energy transition.

% CP1
The first paper presents an investigation on the relation between critical moments in an operational power system model, defined through energy not served, and large scale weather patterns based on the available 28 years of historical weather and 12 scenarios of a future European power system.
It highlights the relation between structural shortage and some of the prevailing weather patterns in the winter period.

% Ext1
The second paper presents a post-processing method to investigate the effect of climate change on the adequacy of the European power system, and as it shows a significant effect, it highlights the importance of incorporating climate change in adequacy assessments.

%%%%%%%%%%%%%%%%%%%%%%%%%%%%%%%%%%%%%%%%%%%%%%%%%%
%%%% ~~~~ Part IV Synthesis
\subsection*{Synthesis and practical application}
% Conclusion
In this thesis a number of data driven methods to increase the understanding of energy-meteorological variability and its impact on energy system operations are shown.
By studying the existing algorithms, methods and models from the overlapping perspective of an energy-, climate-, and data scientist, transdisciplinary approaches have been developed to provide the understanding needed for grounded decisions by researchers and stakeholders in the power sector.

% practical application


% interdisciplinarity
The presented results within this thesis are the product of an inter- and transdisciplinary research effort, and the findings of this thesis are as interconnected as the disciplines that informed them.
%%%%%% JOKE ON BUZZ-WORDINESS

