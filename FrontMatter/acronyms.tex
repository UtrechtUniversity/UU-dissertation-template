%%%%%%%%%%%%%%%%%%%%%%%%%%%%%%%%%%%%%%%%%%%%%%%%%%
%%%%		~~~~ EDITOR'S NOTE ~~~~
%%%%%%%%%%%%%%%%%%%%%%%%%%%%%%%%%%%%%%%%%%%%%%%%%%
%
% This file contains the fancy definition of acronyms
%                       - requires a full compilation cycle to work properly
%                       - when defined here, the handle can be used in text with \ac{<handle>}
%
% This is the PhD thesis of Laurens P. Stoop in LaTeX
%                       - Comments, remarks and more: email me at laurensstoop@protonmail.com
%
%%%%%%%%%%%%%%%%%%%%%%%%%%%%%%%%%%%%%%%%%%%%%%%%%%
%%%%	 	 ~~~~ END OF: EDITOR'S NOTE ~~~~
%%%%%%%%%%%%%%%%%%%%%%%%%%%%%%%%%%%%%%%%%%%%%%%%%%



%%%%%%%%%%%%%%%%%%%%%%%%%%%%
%%%%            ~~~~ Example ~~~~
%%%%%%%%%%%%%%%%%%%%%%%%%%%%
%
%
% For a given handle, an acronym's full definition can be provided with:
%        \DeclareAcronym{<handle>}{
%                short=<acronym>,
%                long=<full definition>,
%        }
%
% In text then used with \ac{<handle>}
%       - Full form first time (forced with \acf{<handle>} )
%       - Short form after this (forced with \acs{<handle>} )
%       - Long form  forced with \acl{<handle>}
% If needed specifically

%% Example used in the Introduction
\DeclareAcronym{eu}{
        short=EU,
        long=European Union
}

