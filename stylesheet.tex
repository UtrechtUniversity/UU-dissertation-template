%%%%%%%%%%%%%%%%%%%%%%%%%%%%%%%%%%%%%%%%%%%%%%%%%%
%%%%		~~~~ EDITOR'S NOTE ~~~~
%%%%%%%%%%%%%%%%%%%%%%%%%%%%%%%%%%%%%%%%%%%%%%%%%%
%
% This file contains the definition of the stylesheet used within the dissertation class
%                       -
%
% This is the PhD thesis of Laurens P. Stoop in LaTeX
%                       - Comments, remarks and more: email me at laurensstoop@protonmail.com
%
%%%%%%%%%%%%%%%%%%%%%%%%%%%%%%%%%%%%%%%%%%%%%%%%%%
%%%%	 	 ~~~~ END OF: EDITOR'S NOTE ~~~~
%%%%%%%%%%%%%%%%%%%%%%%%%%%%%%%%%%%%%%%%%%%%%%%%%%



%%%%%%%%%%%%%%%%%%%%%%%%%%%%
%%%%            ~~~~ Stylesheet ~~~~
%%%%%%%%%%%%%%%%%%%%%%%%%%%%

%%%%%%%%%%%%%%%%%%%%%%%%%%%%%%%%%%%%%%%%%%%%%%%%%%
%%%%								~~~~ PREAMBLE ~~~~
%
% In the preamble the following can be found:
%		- Class definitions/call
%		- Extra package invocation
%		- Style definitions
%		- Shortcut code for use throughout the document
%		- \ldots
%
%%%%%%%%%%%%%%%%%%%%%%%%%%%%%%%%%%%%%%%%%%%%%%%%%%

%%%%%%%%%%%%%%%%%%%%%%%%%%%%%%%%%%%%%%%%%%%%%%%%%%
%%%%		Extra package invocation


%% CROP package
% Redefine the behaviour of Crop based on the actual document size and some other mumbo
% Cut marks
% Crop marks
%
%\usepackage[
%% set width and height to a4 width and height + 6mm
%width=21.6truecm, height=30.3truecm,
%% use any combination of these options to add different cut markings
%cam,noinfo,
%% set the type of TeX renderer you use
%pdftex,
%% center the contents
%center
%]{crop}


%% This would allow for opening on left page of chapters
%
% First option
%
%\makeatletter
%\renewcommand*\cleardoublepage{\clearpage\if@twoside
        %        \ifodd\c@page \hbox{}\newpage\if@twocolumn\hbox{}%
        %        \newpage\fi\fi\fi}
%\makeatother
%
% Other option
%
%\csname @openrightfalse\endcsname

%% For access to the lightnigbolt sign
\usepackage{wasysym}

%% For access to written out month/year information
\usepackage{datetime}
\newdateformat{monthyeardate}{\monthname[\THEMONTH] \THEYEAR}

%% Create a list of acronyms.
\usepackage{acro}
% Set the acronyms in the PDF to link back to the acronym list
\acsetup{make-links=true}

\usepackage{layout}
\usepackage{tocloft}
\usepackage[parfill]{parskip}
\usepackage[T1]{fontenc}
\usepackage{subcaption}
\usepackage{setspace}
\usepackage{import}
%\usepackage[final]{changes}
\usepackage{tikzscale}
\usepackage{tikz}
%\usepackage{dblfloatfix}
\usepackage{lipsum}
\usepackage{mdframed}

\usepackage[many]{tcolorbox}

% Package to provide a way to review
%       - gives TeXstudio access to the tools
%\usepackage{easyReview}


% Definition of the bibliography and related packages
\usepackage[square,numbers,sectionbib]{natbib}
\usepackage{chapterbib}

\usetikzlibrary{shapes,arrows,chains}

\usepackage{pgfplots}
\pgfplotsset{compat=newest}
%\pgfplotsset{plot coordinates/math parser=false}
\newlength{\fwidth}
\colorlet{lcnorm}{black}


%\setlength{\oddsidemargin}{-13pt}
%\setlength{\evensidemargin}{-10pt}
\setlength{\voffset}{18pt}

%\DeclareSIUnit{\sqrthz}{\ensuremath{\sqrt{\text{\hertz}}}}



% Layout settings:
\renewcommand{\headrulewidth}{1.5pt}% 2pt header rule

\tikzset{font=\small}

\pgfplotsset{compat=1.8}

%% Command to add a numbered tag to an equation
\newcommand\numberthis{\addtocounter{equation}{1}\tag{\theequation}}


%% Small change to plot all figures with fbox, for better layout
\LetLtxMacro\latexincludegraphics\includegraphics % save the meaning of \includegraphics

% pass the image to \shadowbox
%\renewcommand{\includegraphics}[2][]{\fbox{\latexincludegraphics[#1]{#2}}}

%% Change the behavior of subscript to textmode (without the amsmath spacing)
\begingroup\lccode`~=`!
\lowercase{\endgroup\def~}#1{_{\mathrm{#1}}}

\AtBeginDocument{\mathcode`!=\string"8000 }

%%% Fixing hyphenation issues:
\hyphenation{im-pe-dance}
\setlength{\marginparwidth}{0pt}

%% Create a macro to easily draw grids for help in positioning text
\newcommand{\Grid}{
	\draw[step=1.0,red,thin,xshift=1cm,yshift=1cm] (-6,-6) grid (4,4);
	\draw[black,thick] (-5,0) grid (5,0);
	\draw[black,thick] (0,-5) grid (0,5);
}





%% TOC
\makeatletter

\newcommand{\chaptoc}{
	\vspace{0.6cm}
	\startcontents[chaps]
	\begin{tcolorbox}[
		colframe=thumb\arabic{thumbcounter},
		width=\linewidth,
		enhanced,
		top=10pt,
		bottom=10pt,
		nobeforeafter,
		outer arc=0pt,
		arc=0pt,
		boxrule=1.2pt,
		colback=white,
		overlay={
			\node[anchor=west,fill=white,inner xsep=6pt, text=thumb\arabic{thumbcounter}]
			at ([xshift=10pt]frame.north west)
			{\textbf{Contents}};
		}
		]
		\printcontents[chaps]{}{1}{}
	\end{tcolorbox}
	\vspace{2cm}}


%% Make changes to the part so that the contets end up on the titlepages
\let\LaTeXStandardPart\part%
\newcommand{\unstarredpart@@noopt}[1]{%
	\unstarredpart@@opt[#1]{#1}%
}%

\newcommand{\unstarredpart@@opt}[2][]{%
	\cleardoublepage% (For clearing content before!!!)
	\begingroup%
	\let\newpage\relax%
	\LaTeXStandardPart[#1]{#2}%
	\endgroup%
}%

\newcommand{\starredpart}[1]{%
	\LaTeXStandardPart*{#1}%
}%

\newcommand{\unstarredpart}{%
	\@ifnextchar[{\unstarredpart@@opt}{\unstarredpart@@noopt}%
}%

\renewcommand{\part}{%
	\@ifstar{\starredpart}{\unstarredpart}%
}%

\makeatother










%% Define the toc colors for the parts, and add the horizontal lines
\renewcommand{\cftpartfont}{\hypersetup{linkcolor=thumb\arabic{colorcounter}}}
\renewcommand{\cftpartpresnum}{\hypersetup{linkcolor=thumb\arabic{colorcounter}}}

\titlecontents{part}%
[0pt]{\stepcounter{colorcounter}\color{thumb\arabic{colorcounter}}\bfseries\large\protect\addvspace{10pt}\titlerule[1pt]\addvspace{1.3ex}}
{}{\partname~}
{\hfill\contentspage}%
[\addvspace{0.7ex} {\titlerule[1pt]} \addvspace{5pt}]%

\renewcommand{\cftchapfont}{\bfseries\hypersetup{linkcolor=thumb\arabic{colorcounter}}}
\renewcommand{\cftchappagefont}{\bfseries\color{thumb\arabic{colorcounter}}}

\renewcommand{\thepart}{\color{thumb\arabic{colorcounter}}\Roman{part}} % Adjust the color of the part as well

\DeclareRobustCommand{\IncCLR}{\protect\stepcounter{colorcounter}}

\newcounter{thumbcounter}
\newcounter{colorcounter}